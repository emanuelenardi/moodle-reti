\question[1]
WiFi, ovvero 802.11, è un protocollo di livello MAC basato su:

\begin{checkboxes}
	\choice CSMA/CA, ovvero Carrier Sense Multiple Access with Collision Avoidance, dove CA significa che quando le stazioni rilevano il canale occupato restano in ascolto e, nel momento in cui il canale si libera estraggono una variabile casuale per decidere se trasmettere immediatamente oppure rimandare la trasmissione in un tempo futuro scelto a caso (p-persistenza).

	\choice CSMA/CA, ovvero Carrier Sense Multiple Access with Collision Avoidance, dove CA significa che esiste una procedura di risoluzione delle contese che evita le collisioni.

	\CorrectChoice CSMA/CA, ovvero Carrier Sense Multiple Access with Collision Avoidance, dove CA significa che quando le stazioni rilevano il canale occupato restano in ascolto e, nel momento in cui il canale si libera effettuano una procedura di backoff con una finestra di contesa stabilita a priori e pari a un opportuno CWmin, ciascuna stazione estrae una variabile causale di backoff in questa finestra.

	\choice CSMA/CA, ovvero Carrier Sense Multiple Access with Collision Avoidance, dove CA che viene implementato un handshake preliminare per evitare il problema del terminale nascosto.
\end{checkboxes}
