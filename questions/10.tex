\question[1]
I protocolli di routing di tipo Distance Vector si basano sullo scambio periodico delle tabelle di instradamento dei router con i router adiacenti. Identificare l'affermazione \emph{falsa} tra le seguenti.

\begin{checkboxes}
	\choice Un router invia periodicamente il proprio Distance Vector ai vicini anche se non c'è alcuna modifica nelle distanze verso tutte le destinazioni.

	\choice Quando un router riceve un vettore delle distanze da un vicino ricalcola il costo (distanza) necessario per raggiungere le destinazioni contenute nel vettore appena ricevuto e se il costo minimo calcolato per raggiungere tutte le destinazioni non è diminuito non effettua alcuna azione.

	\CorrectChoice Quando un router riceve un vettore delle distanze da un vicino ricalcola il costo (distanza) necessario per raggiungere le destinazioni contenute nel vettore appena ricevuto e utilizza come "next-hop" il vicino che ha inviato il vettore stesso.

	\choice Quando un router riceve un vettore delle distanze da un vicino ricalcola il costo (distanza) necessario per raggiungere le destinazioni contenute nel vettore appena ricevuto e se il costo calcolato per raggiungere qualcuna delle destinazioni è diminuito utilizza come "next-hop" per queste destinazioni il vicino che ha inviato il vettore stesso e manda il proprio Distance Vector a tutti i vicini.
\end{checkboxes}
